\chapter{Introduction}

\subsection{Evolution is the Driving Force of Biological Diversity}

One of the most striking aspects of life is the vast diversity of
forms and functions displayed by living things. Organisms are beautifully
adapted to a broad range of environments and life styles; from bacteria
that thrive in deep see thermal vents \citep{zierenberg_life_2000},
to plants that live in high alpine meadows \citep{bliss_adaptations_1962},
to exquisitely colorful poisonous frogs that roam the rainforest \citep{summers_evolution_2001}.
With such diversity on display it is easy enough to forget that all
organisms on Earth share a common ancestor in the distant past \citep{darwin_origin_1859,weiss_physiology_2016,glansdorff_last_2008,lane_how_2010}.
Over unfathomable stretches of time, life has diversified from that
common ancestor into the amazingly complex and dynamic biosphere that
is familiar to us today. Perhaps the most incredible facet of this
diversity is that it is produced via the stochastic process of evolution
\citep{darwin_origin_1859,lewontin_genetic_1974,jacob_evolution_1977,woese_towards_1990,nei_molecular_1998}.
How this random process generates the rich biology observed on Earth
is the primary driving question of evolutionary biology. 

Evolution occurs via the change of heritable traits over the course
of many generations of organisms. The process acts on traits that
are displayed in some way at the macroscopic, organismal level \citep{lewontin_genetic_1974,gillespie_population_1998,gillespie_molecular_1984,fisher_genetical_nodate,wright_evolution_1931}.
However, at the heart of trait heritability are the genes that encode
traits at the genetic level. The projection of underlying genetics
into phenotypes is commonly referred to as the genotype-phenotype
map \citep{fontana_shaping_1998,stadler_genotype-phenotype_2006}.
Evolutionary processes such as natural selection and genetic drift
drive the fixation of mutant genes that lead to new traits in populations
\citep{gillespie_simple_1983,gillespie_molecular_1984,gillespie_population_1998,orr_population_2002}.
Over time this fixation process can lead to substantial changes in
the genetic makeup of a population of organisms and result in the
formation of new species with different organism-level traits \citep{mcniven_genetic_2011,abbott_hybridization_2013,hedges_tree_2015,egan_experimental_2015,fisher_genetical_nodate}. 

Most functionally-important genes encode proteins, which are the workhorses
of molecular processes in living organisms \citep{alberts_protein_2002,whitford_proteins:_2005}.
They catalyze chemical reactions, form the basis of structural scaffolds,
transport ions and small molecules, act as signals, and regulate the
function and production of other molecules \citep{alberts_protein_2002,whitford_proteins:_2005}.
The emergent outcome of all the intertwining protein roles is ultimately
manifested in the macroscopic phenotype of an organism. The vast array
of protein functions necessary to construct organisms requires a large
diversity of proteins, all of which are encoded by genes and integrated
into the broader system. This framework imposes an extremely complex
set of constraints that govern the way in which new traits---that can
be “seen” by evolution---can be achieved. Thus, a molecular-level understanding
of evolution is critical to understanding the process on larger scales. 

\subsection{Evolutionary Biochemistry is a Powerful Tool for Understanding Biology}

Most studies in traditional evolutionary biology have focused on understanding
the genetics of evolution. Genetics provides a very useful tool to
dissect the basis of evolutionary logic at the level of encoding architecture.
A genetic framework has also been critical for developing a population-level
understanding of evolution and creating useful mathematical models
of evolutionary processes \citep{fisher_genetical_nodate,wright_evolution_1931,kimura_number_1964,crow_mathematics_1970,gillespie_simple_1983,gillespie_molecular_1984,nei_relative_1988,gillespie_population_1998,orr_population_2002,orr_probability_2005}.
These models have made it possible to make predictions that can be
tested experimentally, thus furthering our ability to understand evolutionary
dynamics and outcomes \citep{hohenlohe_using_2010,romiguier_comparative_2014,mackay_epistasis_2014,tiffin_advances_2014,huang_population_2014,lynch_mutation_2016}.
However, the rules that ultimately govern the inner workings of biological
organisms are those of physics and chemistry \citep{berg_random_1993,sella_application_2005,dill_molecular_2010,kondo_reaction-diffusion_2010,dill_physical_2011,ghosh_role_2016}.

Although the phenomenonological genetic ``laws'' that govern evolutionary
processes are now largely understood, it is unclear how they are connected
to the physical laws that govern the universe. This disconnect is
one of the most prominent barriers to understanding molecular evolution.
To truly understand how evolution works at the molecular level this
relationship must be determined. The need to understand how physicochemical
principles shape evolutionary outcomes has spawned the field of evolutionary
biochemistry. This field seeks to understand the evolution of molecular
phenotypes at the biochemical level and to relate the molecular phenotypes
to implications for evolution at larger scales \citep{feeney_evolutionary_1969,harms_evolutionary_2013,harms_analyzing_2010}.
Many pressing questions remain unanswered. Are there general evolutionary
trends in biochemical features over very long time scales? How robust
are protein functions to alterations in amino acid coding sequence
and how does this robustness affect the maintenance of important traits
during evolution? How do protein copies evolve after they are generated
by gene duplication events? How do correlations between mutations
in protein sequences shape evolutionary possibilities? Can we understand
large-scale evolutionary processes in terms of simpler molecular-level
constraints and rules? These questions are unified by the broader
inquiry: how do physical rules shape the genotype-phenotype map? 

The field of evolutionary biochemistry has rapidly expanded since
its inception and provided a great deal of insight into evolution
at the molecular level. A critical workhorse of evolutionary biochemistry
has been ancestral sequence reconstruction (ASR) \citep{pauling_chemical_1963,harms_evolutionary_2013,harms_historical_2014,wheeler_thermostability_2016}.
ASR is a statistical technique that utilizes a molecular phylogeny
to infer the sequences of ancestral nodes \citep{liberles_ancestral_2007,hanson-smith_robustness_2010,harms_evolutionary_2013,eick_robustness_2017}.
This technique has allowed many researches to directly assess ancestral
protein activities using biochemical experiments, making it extremely
powerful for characterizing evolutionary history \citep{harms_analyzing_2010,harms_evolutionary_2013,harms_historical_2014,wheeler_thermostability_2016}.
In some cases, entire evolutionary trajectories---composed of historical
substitutions---have been reconstructed \citep{bridgham_epistatic_2009,mckeown_evolution_2014,boucher_atomic-resolution_2014}.
Relationships between protein structure, function, and evolutionary
history have been characterized for a wide variety of proteins \citep{boucher_atomic-resolution_2014,anderson_intermolecular_2015,clifton_ancestral_2016,mckeown_evolution_2014,hart_thermodynamic_2014,harms_historical_2014,aakre_evolving_2015,eick_evolution_2012,wilson_using_2015,risso_hyperstability_2013}.
Much has been learned about how biochemistry and biophysics constrain
and shape protein evolution. Furthermore, evolutionary approaches
have been used---with great success---to winnow the substitutions observed
in extant proteins down to those that are important for a given biochemical
function. For example, ASR was used to identify residues that are
important for binding selectivity of the drug Gleevec by Ab1 and Src
kinases \citep{wilson_using_2015}. 

Detailed biochemical studies have also helped to clarify the importance
of phenomena such as epistasis---the non-additivity of mutations \citep{harms_historical_2014,mackay_epistasis_2014,starr_epistasis_2016,sailer_detecting_2017}---and
pleiotropy---in which proteins have roles in multiple distinct biological
processes \citep{wolf_contribution_2006,wagner_gene_2008}---in determining
evolutionary outcomes. These effects can reduce the evolutionary degrees
of freedom allowed for a protein and result in effects such as historical
contingency \citep{blount_historical_2008,harms_historical_2014}.
For example, to evolve specificity for a new hormone ligand the glucocorticoid
receptor required a permissive subtitution that alleviated the results
of an otherwise deleterious functional substitution in the ligand
binding site \citep{harms_historical_2014}. Studies that incorporate
biochemical and functional work have futher demonstrated that the
broader systemic architecture of the cellular environment can constrain
the mechanisms by which biochemical changes underly organismal phenotypes
\citep{des_marais_escape_2008,smith_gene_2011,smith_functional_2013,sorrells_intersecting_2015}.
Certain systems have far greater constraints on the allowed biochemical
changes. For example, the evolution of new flower colors in plants
often requires both functional amino acid substitutions and regulatory
changes, but the genes that are subject to these different types of
changes vary depending on pleiotropic consequences \citep{streisfeld_altered_2009,streisfeld_genetic_2009,wessinger_lessons_2012,wessinger_predictability_2014}. 

Evolutionary biochemistry has provided great insight into the molecular
mechanisms of evolution. However, there is a key limitation that is
prevalent in most previous work. Evolutionary biochemical studies
have focused almost exclusively on proteins that exhibit very rigidly
defined biochemical features. For example, the evolution of binding
specificity has largely been studied in proteins such as enzymes and
transcription factors that exhibit exquisite binding specificity for
targets \citep{zarrinpar_optimization_2003,weinreich_darwinian_2006,copley_toward_2012,reinke_networks_2013}.
These studies have revealed key patterns in the evolution of specificity,
such as consistent occurrence of subfunctionlization and neofunctionalization
following gene duplications \citep{boucher_atomic-resolution_2014,eick_evolution_2012,mckeown_evolution_2014,hudson_structure_2014,howard_ancestral_2014}.
Similarly, studies of proteins binding to other biologically-relevant
targets such as metal ions have traditionally considered very well-defined
coordination systems, like those found in metalloproteases and Zinc
finger proteins \citep{yannone_metals_2012}. However, many proteins
do not exhibit such exquisite such exquisite biochemical properties
\citep{ekman_what_2006,uchikoga_specificity_2016,bhattacharya_target_2004,chin_calmodulin:_2000,mitchell_evolutionary_2013,gfeller_multiple-specificity_2014,schreiber_protein_2011,copley_evolutionary_2015}.
A large number of proteins bind to targets with low specificity and
limited binding-site conservation. The biological relevance of the
biochemical properties of these proteins is less well understood.
It is thus unclear how well evolutionary studies of typical protein
model systems translate to the broad array of proteins with plastic
biochemical properties. 

\subsection{The S100 Protein Family is a Useful Model System to Probe the Evolution
of Low-specificity Proteins}

This dissertation focuses on case studies in evolutionary biochemistry
that address unanswered questions in molecular evolution. Chapter
II consists of a literature review addressing the evidence for global
trends in protein evolution over very long time scales. The remaining
studies are unified by questions surrounding the evolution of protein-target
interactions in proteins that have labile binding interfaces and/or
highly-variable binding partners. Each case study dissects a specific
aspect of evolution at the molecular level. The studies use a combination
of experiments and computational analysis methods to address how biochemistry
relates to broader questions in evolutionary biology. 

Chapters III, IV, and V of this dissertation make extensive use of
the S100 proteins as an experimental model system, which warrants
an introduction to the protein family. The S100s are a large family
of small, calcium-dependent signaling proteins \citep{donato_functions_2013,donato_intracellular_2003,zimmer_evolution_2013,heizmann_new_nodate,chazin_relating_2011}.
The proteins are generally homodimeric and transduce signals via a
calcium-ion driven conformational change \citep{santamaria-kisiel_calcium-dependent_2006,donato_functions_2013}.
The family originated at the base of the Metazoan lineage and subsequently
diversified over several hundred million years \citep{kraemer_structural_2008,zimmer_evolution_2013,wheeler_multiple_2016}.
Mammals possess approximately thirty S100 genes including those encoding
fusion proteins, in which the S100 acts as a single domain inside
a larger domain architecture \citep{zimmer_evolution_2013,wheeler_multiple_2016,kizawa_s100_2011,gutknecht_identification_2017,contzler_cornulin_2005}.
S100 proteins play a wide array of biological roles inside and outside
of cells; including inflammatory signaling \citep{marenholz_s100_2004,leclerc_binding_2009},
regulation of cell proliferation \citep{riuzzi_s100b_2011,zhu_s100a16_2016,cho_pentamidine_2016},
antimicrobial activity \citep{damo_molecular_2013,hayden_high-affinity_2013},
and control of apoptosis in some cell types \citep{tsoporis_s100b_2005,tsoporis_expression_2008}.
The diversity of functions performed by the S100s is perhaps surprising
considering the small size of the proteins, overall similarity of
S100 amino acid sequences, and conservation of the folded form. However,
the proteins have evolved an array of useful biochemical features
that aid in carrying out biological functions. The proteins possess
the ability to bind both calcium ions and other metal ions. Calcium-induced
conformational changes result in the exposure of a hydrophobic path
on the S100 dimer surface, which facilitates binding of target proteins
\citep{santamaria-kisiel_calcium-dependent_2006,chazin_relating_2011}.
The specificity of these hydrophobic binding sites varies among members
of this family, although it has not been systematically studied prior
to the work in this dissertation \citep{chazin_relating_2011,wafer_novel_2013}.
It is sometimes presumed that this biochemical specificity contributes
to the biological specialization of the S100s \citep{chazin_relating_2011}.
This notion is supported by the fact that only some S100s are capable
of binding to certain target proteins. For example, many S100s bind
to and activate the inflammatory RAGE protein, but this not a universal
trait of the family \citep{leclerc_binding_2009,cho_pentamidine_2016}.
However, it has also been proposed that most functional specificity
of S100 proteins is acheived by control of differential expression
\citep{donato_functions_2013,donato_intracellular_2003,marenholz_s100_2004}. 

The biological importance of binding to metal ions other than calcium---which
occurs at different ion binding sites---has not been well studied. This
dissertation primarily uses the S100s as a model to address evolutionary
questions, because they possess a rich evolutionary history, diverse
biochemical features, and exhibit low specificity for interaction
partners. However, the evolutionary biochemical work presented in
chapters III, IV, and V also sheds light on biologically-relevant
aspects of the S100 family. The work provides several opportunities
and resources for more biologically-oriented future studies. 

\subsection{Chapter-by-chapter Breakdown of Dissertation}

Chapter II comprises a literature review---co-written with Shion An
Lim (SAL), Susan Marqusee (SM), and my advisor Michael J. Harms (MJH).
The review addresses the question of whether or not proteins display
global evolutionary trends over very long time scales. Two case studies
are used as key examples of hypothesized trends: the gradual reduction
of protein thermostability due to cooling of the Earth and the gradual
increase in protein binding specificity due to continued specialization
in ever-more-complex proteomes. Based on a thorough summary of evolutionary
biochemistry literature, there does in fact appear to be some evidence
for a gradual decline of thermostability on the billions-of-years
time scale. However, there are still relatively few studies that probe
this question. A more substantial body of evidence will need to be
accumulated to make a strong argument for a global trend. The need
for more experimental evidence is even more pronounced with regard
to the question of broad trends in specificity. There are few studies
that have addressed this question directly, and none to date that
have done so using a truly unbiased experimental approach. This chapter
provides an overview of the idea that there are global trends in protein
evolution, makes a strong case that further experimental studies are
needed to resolve the ongoing debates on this topic, and suggests
strategies and experiments to maximize the current understanding in
the field. The literature review presented in this chapter was published
in the journal Current Opinions in Structural Biology \citep{wheeler_thermostability_2016}.

Chapter III probes the evolutionary lability of a biologically-important
biochemical feature. The S100 protein family is used as a model system
to address this question. The phylogenetic history of the S100s is
reconstructed to yield the highest-quality phylogeny of the S100 family
to date. The history of transition metal binding in the S100s is then
traced by mapping the results of detailed in vitro measurements of
metal-ion binding onto this high-quality phylogeny. These results
show that binding of transition metals is conserved across almost
the entire S100 family, a more universal result than any previous
study. By using mutagenesis studies it is further established that
not all S100 proteins use the same amino acids or even the same site
to bind metal ions. The binding of metal ions to a very early branching
S100 protein is measured for the first time, which demonstrates that
binding of transition metals is an ancestral feature of the S100 protein
family. The results of this chapter speak to the surprising level
of lability---at the amino acid level---of S100 protein metal binding
sites; highlighting the fact that an ancestral molecular phenotype
can be maintained at the overall level of behavior even while the
underlying biochemical basis fluctuates over evolutionary time. The
work in this chapter has been published as a research article in PLoS
One, co-authored with Micah T. Donor (MTD), James S. Prell (JSP), and
Michael J. Harms (LC Wheeler is the first author) \citep{wheeler_multiple_2016}. 

Chapter IV delves further into the biophysics of metal binding in
one particular member of the S100 protein family, S100A5. Little is
known about the biological roles of S100A5. A previous publication
indicated that the protein exhibits antagonism between the binding
of Ca\textsuperscript{2+} and Cu\textsuperscript{2+} ions. This
feature is unique amongst S100 proteins and has been considered one
of the key features of S100A5. Proposed biological roles for the protein
typically involve Ca\textsuperscript{2+}/Cu\textsuperscript{2+}
antagonism. In chapter IV, it is demonstrated that antagonism between
the binding of Ca\textsuperscript{2+} and Cu\textsuperscript{2+}
is likely an artifact of the experiments done in the original study.
Instead, it is shown that S100A5 can bind Ca\textsuperscript{2+}
and Cu\textsuperscript{2+} independently, which changes the biological
implications of metal binding to the protein. Furthermore, this chapter
adds to the evolutionary story of metal binding by demonstrating another
unique biochemical modification that has evolved in the S100 family.
The work in this paper is currently in press as a research
article in the journal BMC Biophysics, co-authored with Michael J.
Harms. 

Chapters V and VI address the evolution of binding specificity in
two proteins following gene duplication from a common ancestor. Again,
the S100 protein family proves to be a useful model system to address
this question. The proteins S100A5 and S100A6 arose from a duplication
approximately 300 million years ago. They subsequently evolved to
have different protein-binding specificity, distinct expression patterns,
and perform different cellular roles. Despite having distinct specificity,
both proteins can be described as “sloppy” or having very low biochemical
specificity. Previous studies have addressing the question of evolving
specificity have used highly-specific proteins and small sets of known
binding partners that are biased by a priori knowledge. For these
reasons, previous studies are limited in understanding the evolution
of specificity in low-specificity proteins. The sloppiness of S100s
makes them an excellent system to study how binding specificity evolves
in an inherently noisy low-specificity system. 

Chapter V comprises a biochemical study of the evolution of peptide
binding specificity in the S100A5-S100A5 clade. The oldest ancestor
of S100A5 and S100A6 is resurrected using ancestral sequence reconstruction
(ASR). Detailed calorimetric measurements of binding to a small set
of peptide targets are then used to compare specificity across a set
of orthologous and paralagous S100A5 and S100A6 proteins. It is demonstrated
that peptide binding is driven primarily by the hydrophobic effect
and that specificity is readily changed by the addition of mutants
into the peptide binding interface. Furthermore, this work reveals
that the specificity of S100A5 and S100A6 have undergone an apparent
pattern of subfunctionilization. This result is striking, because
it demonstrates that proteins with very low biochemical specificity
can undergo similar patterns of evolution to proteins with high specificity.
The work in this chapter is in review as a research article
in the journal Biochemistry, co-authored with Jeremy A. Anderson (JAA),
Anneliese J. Morrison (AJM), Caitlyn E. Wong (CEW), and Michael J.
Harms. The submitted article has also been uploaded to the preprint
server BioArxiv \citep{wheeler_conservation_2017}. 

Chapter VI introduces new experimental and analysis pipelines for
studying the evolution of specificity. An unbiased high-throughput
approach, incorporating phage display and deep sequencing, is used
to measure the binding of a large random peptide library to human
S100A5, human S100A6, and the last common ancestor. Strikingly, the
pipeline uncovers the lack of sequence-based rules that govern binding
preferences of the S100 proteins. Instead, preferences appear to be
defined by general physicochemical features of the peptide targets
that can be used to generate a predictive model. The pipeline reveals
overall patterns in the evolution of specificity along the S100A5
and S100A6 lineages. S100A5 exhibits a strong signal of subfunctionilization,
while S100A6 appears to differ little from the ancestor. This chapter
highlights the importance of using unbiased approaches to study the
evolution of specificity and speaks to the necessity of understanding
different classes of protein features when probing molecular evolution.
The work in this paper is being prepared as a research article that
will be submitted to the journal MBE, co-authored with Michael J.
Harms. 

\subsection{Broader Impacts}

The studies described in this dissertation contribute the broader
evolutionary biochemistry literature by addressing a set of topics
that have remained ambiguous. There has been a lack of studies addressing
the evolution of biochemical features in proteins that have highly
diverse sets of binding partners. Much of the experimental basis for
understanding evolution of protein binding specificity has instead
been based on proteins with exquisite specificity profiles \citep{zarrinpar_optimization_2003,stiffler_pdz_2007}.
For example, enzymes, receptors, and transcription factors that have
well-defined chemical binding preferences are workhorses of evolutionary
biochemistry studies \citep{harms_historical_2014,boucher_atomic-resolution_2014,mckeown_evolution_2014,eick_evolution_2012}.
The work presented in the following chapters probes key aspects of
the evolution of binding specificity in proteins without such obvious
rules. The S100 proteins act as an excellent model system to tackle
these problems, because they have a variety of conserved biochemical
behaviors that have nontheless been labile at the amino acid level
during diversification of the family \citep{wheeler_multiple_2016}.
In particular, the ability of the S100s to bind to a variety of transition
metals with similar affinities, and the ability to bind extremely
diverse short peptide regions of target proteins are used as exemplary
biochemical features. 

Studies on both the binding of metal ions and peptides reveal several
key evolutionary trends that speak to the evolution of biochemical
features in sloppy proteins such as the S100s. 1) a biochemical output---such
as binding of transition metals or peptides with moderate afffinity---can
be acheived and conserved despite extensive variability in amino acid
ligands that form binding sites. 2) Specificity can nontheless be
achieved and conserved in proteins with highly diverse binding partners
and labile binding sites. 3) Evolutionary patterns in proteins with
low biochemical specificity nontheless resemble those observed in
high-specificity proteins. 4) Evolutionary patterns can differ along
duplicate lineages following gene duplication. 5) Unbiased high-throughput
techniques are essential for inferring historical patterns of specificity
in proteins with large diverse sets of binding partners. These observations
contribute substantially to our understanding of what types of biochemical
features are important during the evolution of proteins that do not
meet the criterion of exquisite binding specificity. Despite relaxed
binding rules, flexible binding sites, and highly-diverse binding
partners these proteins nonetheless exhibit evolutionary patterns
that are reminiscent of those the field has come to expect from canonical
examples. This key result suggests that proteins such as the S100s---despite
the variability of there biochemical behaviors---are therefore operating
under similar rules to other proteins. Therefore, proteins with highly
variable binding partners and labile binding sites do not necessarily
represent a fundamentally different class of proteins---subject to special
evolutionary constraints---but rather are similarly constrained by evolutionary
and biochemical forces in a way that can be understood by careful
experimentation.
