\abstract{Proteins perform an incredible array of functions facilitated by a diverse set of biochemical properties.
Changing these properties is an essential molecular mechanism of evolutionary change, with major
questions in protein evolution surrounding this topic. How do new functional biochemical features
evolve? How do proteins change following gene duplication events? I used the S100 protein family as a
model to probe these aspects of protein evolution. The S100s are signaling proteins that play a diverse
range of biological roles binding Calcium ions, transition metal ions, and other proteins. 
Calcium drives a conformational change allowing S100s to bind to 
diverse peptide regions of target proteins. I used a
phylogenetic approach to understand the evolution of these diverse biochemical features. 
Chapter I comprises an introduction to the disseration. 
Chapter II is a co-authored literature review assessing available evidence for global trends in protein evolution. 
Chapter III describes mapping of transition metal binding onto a maximum likelihood
S100 phylogeny. Transition metal binding sites and metal-driven structural changes are a
conserved, ancestral features of the S100s. However, they are highly labile at the amino acid level. 
Chapter IV further characterizes the biophysics of metal binding in the S100A5 lineage, revealing that the oft--cited Ca\textsuperscript{2+}/Cu\textsuperscript{2+} antagonism of S100A5 is likely due to an experimental artifact of previous studies. Chapter V uses the S100 family to investigate the evolution of binding specificity. Binding specificity for a small set of peptides in the duplicate S100A5 and S100A6 clades. Ancestral sequence reconstruction reveals a pattern of clade-level conservation and apparent subfunctionalization along both lineages.
In chapter VI, peptide phage display, deep-sequencing, and machine-learning are combined to quantitatively
reconstruct the evolution of specificity in S100A5 and S100A6. S100A5 has subfunctionalized from the ancestor, while S100A6 specificity has shifted. The importance of unbiased approaches to measure specificity are discussed. This work highlights the lability of conserved functions at the biochemical level, and measures changes in specificity following gene duplication. Chapter VII summarizes the results of the dissertation, considers the implications of these results, and discusses limitations and future directions. 


This dissertation includes both previously published/unpublished and co-authored material.}
