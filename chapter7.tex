\chapter{SUMMARY AND CONCLUDING REMARKS}

\subsection{Contributions to the field of evolutionary biochemistry }

The tools of evolutionary biochemistry have become an important approach
to understanding molecular evolution. This field provides the basis
to determine how the biochemical properties of organisms---at the level
of proteins---shape the evolution of new phenotypes. A broad array of
biologically-relevant molecular-level changes have been characterized,
revealing the importance of biochemical constraints in contributing
to evolutionary changes in signalling \citep{mckeown_evolution_2014},
metabolism \citep{boucher_atomic-resolution_2014}, nutrient transport
\citep{clifton_ancestral_2016}, and many other aspects of biology.
In many cases these changes required biochemical modifications that
altered the recognized binding partners of proteins \citep{eick_evolution_2012,boucher_atomic-resolution_2014,hudson_distal_2015,alhindi_protein_2017,risso_hyperstability_2013,soskine_mutational_2010,carroll_evolution_2008}.
Despite substantial progress in understaning the molecular basis of
evolutionary alterations, there are still a vast number of unsanwered
questions. One limitation of previous studies has been to focus on
proteins with very specific sets of binding partners. Exquisite specificity
is important for biology. However, many proteins exhibit the ability
to interact with highly-diverse binding partners. These proteins are
also critical for many biological processes, but previous work has
shied away from using such proteins as models. The biological roles
and relevance of biochemically-defined sets of binding partners are
less obvious in these cases than in more canonical examples of proteins
with tight specificity. This dissertation focused on helping to close
the gap in understanding how proteins with highly-variable binding
partners and binding sites evolve new biochemical specificity. 

The work presented in this dissertation represents a contribution
to the field of evolutionary biochemistry. It makes important contributions
to understanding the evolution of proteins with diverse biochemical
features, which differ from many other model protein systems in the
variability of their binding partners. The S100 proteins proved to
be a useful model for probing the evolution of binding specificity.
Tracing the evolution of of both metal ions and small peptide binding
by the S100s revealed several key observations. 1) A biochemical output
can be conserved over evolutionary time despite extensive amino acid
turnover in binding sites. 2) Proteins with low biochemical specificity
can nonetheless be subject to evolutionary constraints that maintain
a given specificity profile. 3) The evolutionary patterns that follow
gene duplications in low-specicificty proteins are similar to those
observed in high-specificity proteins. 4) Following gene duplication
the duplicate lineages can undergo differential changes in specificity,
such as subfunctionalization on one lineage and neofunctionalizatin
on the other. 

\subsection{Limitations and future directions}

The work in this dissertation has contributed to understanding how
proteins with highly-variable binding partners evolve new biochemical
specificity. The studies presented here are the first to address this
problem in a systematic way. It paves the way for future studies to
improve upon methods, model systems, and connect what has been learned
to more nuanced questions regarding the evolution of binding specificity. 

Nonetheless, there are limitations to this work that should be considered.
For example, the evolution of metal binding was traced across the
entire S100 protein family. This work revealed two key observations:
1) the overall biochemical output of transition metal can be maintained
despite extreme lability of metal-binding sites, and 2) there is substantial
variation in the structural output of metal binding. These observations
suggested that there has been some degree of biochemical specilization
in the response to metal binding that could potentially have direct
biological consequences. The known roles of transition metal binding
in S100s range from antimicrobial sequestration metals \citep{damo_molecular_2013}
to metal-chaperoning as part of a signalling pathway \citep{sivaraja_copper_2006}.
It seems unlikely that a biochemical ouput conserved across the family
has been maintained for hundreds of millions of years in the absence
of a biological role. However, the roles of transition metal binding
remain unknown for most S100 proteins \citep{moroz_role_2010,gilston_binding_2016,wheeler_multiple_2016}.
The work presented here remains merely suggestive of the biological
relevance of this biochemical behavior. The downstream output of the
S100 biochemistry was not characterized in its natural cellular environment.
Future work should thus focus on understanding what role transition
metal binding is playing in the relevant physiological context. Furthermore,
although this work revealed the extreme variability of binding site
ligands and locations, it nonetheless leaves open the question of
exactly which ligands are used throughout the family. For example,
the Cu\textsuperscript{2+} binding ligands of S100A5 still remain unknown. Future
studies should this also seek to map out the transition metal binding
sites of S100s. Eliminating binding sites \textit{in vivo} via site-directed
mutagenesis will further allow their biology of transition metal binding
to be probed. 

With regard to the studies of peptide binding specificity presented
in this dissertation, an unbiased approach was used to trace the evolutionary
history of peptide binding specificity in S100A5 and S100A6 following
gene duplication. This method allowed the total scope of binding partners
to be estimate for each of the duplicate lineages. Combining the method
with ASR further allowed the evolutionary dimension to be directly
assessed, revealing the historical patterns of changes in specificity
that occurred following duplication. The patterns, when compared to
those observed via a more traditional low-throughput method, highlighted
the importance of using such a global, unbiased approach. The low-throughput
method provided valuable, gold-standard information that clearly demonstrates
conservation of specificity profiles within paralogous lineages. However,
it lacked the resolution to characterize how the size and divsersity
of binding sets had changed during evolution. What appeared to be
a symmetric pattern of subfunctionilzation from a less-specific ancestor
on both the S100A5 and S100A6 lineages was revealed by the high-throughput
approach to be far more nuanced. S100A5 did in fact undergo subfunctionlization,
but S100A6 actually underwent a shift in specificity. In fact, it
is possible that the scope of binding partners for S100A6 actually
increased over evolutionary time. This result clearly shows the superiority
of using an unbiased high-throughput approach to chacterize the evolution
of specificity, which is further accentuated by the very low specificity
of proteins like the S100s. 

The unbiased nature of the approach used to measure specificity is
the key strength of the method. However, this is also one of the key
limiations. The very fact that the approach utilized a random set
of peptide targets divorces it from direct biological implications.
The low-throughput study revealed strong conservation of binding specificity
profiles in duplicate lineage. This results strongly argues that there
is indeed biological relevance for the biochemical specificity of
these proteins, because it seems very unlikely that these profiles
would be maintained purely by chance over 320 million years of evolution
without some sort of selection to maintain the set of binding patners.
This argument is particularly strong considering that specificiy can
be readily altered in these proteins by a single amino acid substitution.
However, the biological implications of biochemical specificity in
S100A5 and S100A6 were not directly assessed and remain firmly in
the realm of speculation. This limitation opens up three key questions
that future studies should seek to address. 1) How does the biologically-realized
set of binding partners compare to the scope of all possible partners
defined by biochemistry? 2) What are the biological forces that winnow
the possible set of partners to the realized set? 3) What is the biological
output of the biochemical specificity that has been conserved for
so long? Answering these questions will provide a more complete picture
of how specificity evolves in the S100s, the forces that shape it,
and the biological implications for evolutionary changes in specificity. 




